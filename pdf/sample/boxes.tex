\documentclass{article}
\usepackage{luacode,luatexbase}
\begin{document}
\begin{luacode*}
local GLYPH_ID = node.id("glyph")

local number_sp_in_a_pdf_point = 65782

function math.round(num)
  return math.floor(num * 1000 + 0.5) / 1000
end

-- width/height/depth of a glyph and the whatsit node
local wd,ht,dp,w

-- head is a linked list (next/prev entries pointing to the next node)
function showcharbox(head)
  while head do
    if head.id == 0 or head.id == 1 then
      -- a hbox/vbox
      showcharbox(head.list)
    elseif head.id == GLYPH_ID then
      -- Create a pdf_literal node to draw a box with the dimensions
      -- of the glyph
      w = node.new("whatsit","pdf_literal")
      wd = math.round(head.width  / number_sp_in_a_pdf_point)
      ht = math.round(head.height / number_sp_in_a_pdf_point)
      dp = math.round(head.depth  / number_sp_in_a_pdf_point)

      -- draw a dashed line if depth not zero
      if dp ~= 0 then
        w.data = string.format("q 0.2 G 0.1 w 0 %g %g %g re S f [0.2] 0 d 0 0 m %g 0 l S Q",-dp,-wd,dp + ht,-wd)
      else
        w.data = string.format("q 0.2 G 0.1 w 0 %g %g %g re S f Q",-dp,-wd,dp + ht,-wd)
      end
      -- insert this new node after the current glyph and move pointer (head) to
      -- the new whatsit node
      w.next = head.next
      w.prev = head
      head.next = w
      head = w
    end
    head = head.next
  end
  return true
end

luatexbase.add_to_callback("post_linebreak_filter",showcharbox,"showcharbox")
\end{luacode*}

\section{Heading1}
This is some normal text in heading level 1.
\subsection{Heading2}
This is some normal text in heading level 2
\subsubsection{Heading3}
This is some random text in heading level 3. 
\subsection{H2}
Just another example subsection
\subsubsection{H3}
An subsubsection are full of random text. 


\end{document}
